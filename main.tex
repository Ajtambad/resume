% Document class and font size
\documentclass[a4paper,9pt]{extarticle}

% Packages
\usepackage[utf8]{inputenc} % For input encoding
\usepackage{geometry} % For page margins
\geometry{a4paper, margin=0.5in} % Set paper size and margins
\usepackage{titlesec} % For section title formatting
\usepackage{enumitem} % For itemized list formatting
\usepackage{hyperref} % For hyperlinks

% Formatting
\setlist{noitemsep} % Removes item separation
\titleformat{\section}{\large\bfseries}{\thesection}{1em}{}[\titlerule] % Section title format
\titlespacing*{\section}{0pt}{\baselineskip}{\baselineskip} % Section title spacing

% Begin document
\begin{document}

% Disable page numbers
\pagestyle{empty}

% Header
\begin{center}
\textbf{\huge Amogh Jagadish Tambad}\\[2pt] % Name
tambadamogh@gmail.com \hspace{1mm} $|$ \hspace{1mm} (602) 561-8944  \hspace{1mm} $|$ \hspace{1mm} $linkedin.com/in/amogh-jagadish-tambad$ \hspace{1mm}
\end{center}

% Education Section
\section*{EDUCATION}
\noindent
\textbf{Master of Science}, Computer Science \hfill May 2025 \\ % University name and location
Arizona State University
\hfill GPA: 3.78/4\\ % Degree and GPA
\textbf{Relevant Coursework:} Perceptual Reasoning \& Symbol Grounding, Knowledge Representation and Reasoning, Cloud Computing, Data Visualization, Information Assurance \& Security.\\

\noindent
\textbf{Bachelor of Technology (B.Tech)}, Computer Science and Engineering \hfill May 2021 \\ % University name and location
REVA University, Bangalore, India \hfill GPA: 8.93/10\\ % Degree and GPA
\textbf{Relevant Coursework:} Data Structures and Algorithms, Computer Architecture, Operating Systems.

% Skills Section
\section*{SKILLS}
\begin{itemize}
    \item \textbf{Programming:} Python, C++, Shell Scripting, C, SQL, HTML, Java, JavaScript. % Programming skills
    \item \textbf{Tools and Technologies:} AWS, Git, Jenkins, Heroku, Splunk, Zabbix, Microsoft Excel, Docker, Oracle, MongoDB. % Software skills
    \item \textbf{Libraries and Frameworks:} PyTorch, Flask, OpenCV, Pandas, Keras, scikit-learn.
\end{itemize}

% Experience Section
\section*{EXPERIENCE}
\noindent
\textbf{System Engineer - 1}, Cerner Healthcare
\hfill May 2021 - Jul 2023 % Position and duration
\begin{itemize}
    \item Engaged with the software development team on \textbf{Splunk} upgrades, troubleshooting, and deployments, ensuring up-to-date servers.
    \item Migrated 80\% data from On-prem to \textbf{AWS}, making access to data more flexible, secure, and inexpensive.
    \item Integrated \textbf{Jenkins} and \textbf{GitHub} to maintain important documentation and test merge requests for semantic errors.
    \item Performed \textbf{CI/CD} and 300+ bi-weekly micro-service \textbf{deployments} for web applications to deploy new UI and backend features.
    \item Managed and tracked over 10 projects and 400+ tasks to completion through \textbf{JIRA}, resulting in smooth and error-free delivery. 
\end{itemize}
\noindent
\textbf{Research Intern}, REVA Research \& Development Cell
\hfill Jan 2021 - Apr 2021 % Position and duration
\begin{itemize}
    \item Trained an ML model to estimate insurance costs based on user details by leveraging \textbf{Linear Regression} and \textbf{Random Forest} techniques.
    \item Utilized \textbf{Python} and \textbf{Microsoft Excel} to eliminate unnecessary data and fill up around 120 NaN values. 
    \item Visualized data with graphs and charts made with the \textbf{Matplotlib} library and gained important insights about the impact of lifestyle choices on insurance prices.
    \item Built a web application using \textbf{Flask} and \textbf{HTML}, and deployed it to \textbf{Heroku}, resulting in more accessibility.
\end{itemize}

% Projects Section
\section*{PROJECTS}

\noindent
\textbf{Live Face Recognition Application with AWS}  \hfill Jan 2024 - Present\\ % Project name and location
\textit{Arizona State University, Tempe, Arizona} % Project link and duration
\begin{itemize}
    \item Hosted the application on a hybrid cloud environment with \textbf{AWS} and \textbf{OpenStack}.
    \item Handled concurrent requests using \textbf{Lambda} and \textbf{EC2} instances, while \textbf{S3} and \textbf{DynamoDB} act as data stores.
\end{itemize}

\noindent
\textbf{Novel Object Detection Using Reasoning by Elimination} \hfill Aug 2023 - Dec 2023\\% Project link and duration
\textit{Arizona State University, Tempe, Arizona}
\begin{itemize}
    \item Engineered a system with \textbf{PyTorch} to detect 'novel' objects not included in training data.
    \item Leveraged pre-trained \textbf{ResNet-18} and \textbf{BERT Tokenizer} models to extract features. 
    \item Computed Cosine similarity and trained the agent with \textbf{reinforcement learning} to eliminate seen objects.
    \item Generated an automobile-relevant, real-world dataset of 1260 images using \textbf{Stable Diffusion}. % Project description and contributions
\end{itemize}

\noindent
\textbf{Mask Detection Using CNN and OpenCV} \hfill Apr 2020 - Jun 2020\\ % Project name and duration
\textit{REVA University, Bangalore, India}  
\begin{itemize}
    \item Trained a deep learning model with PyTorch to detect whether a person is wearing a face mask.
    \item Employed Convolutional Neural Networks (\textbf{CNNs}), specifically ResNet-50 architecture to train the model.
    \item Tested the model in real-time with \textbf{OpenCV} by feeding frame-by-frame live video footage to the model.
    \item Obtained a model accuracy of \textbf{92\%} through 10 epochs of training on train and validation data.
    \item Submitted to the 'Jeeva Samrakshak hackathon' organized by REVA university and won \textbf{2nd place}.  % Project description and contributions
\end{itemize}

\noindent
\textbf{Driver Drowsiness Detection System}  \hfill Feb 2020 - Mar 2020\\ % Project name and location
\textit{REVA University, Bangalore, India} % Project link and duration
\begin{itemize}
    \item Built a system capable of detecting driver drowsiness by recording eye and mouth movements.
    \item  Gauged the eye closure duration by utilizing OpenCV and 'frontal face detector' module to distinguish between drowsiness and blinking. 
    \item Accounted for yawning through coordinates near the user's mouth, making the system more robust.
    \item  Integrated an alarm system to alert users of drowsiness, thereby increasing usability. % Project description and contributions
\end{itemize}

% End document
\end{document}
